%---PACKAGES----------------------------------------
\documentclass[a4paper,10pt]{article}

\usepackage{import}
\import{Packages/}{custom_packages.tex}
\import{Packages/}{custom_macros.tex}

\title{\vspace{-2cm}\textbf{Notes symmetries in field theories}}
\author{Louan Mol\\ \textit{Université Libre de Bruxelles}}
\date{Last updated on \today.}

% DOCUMENT -----------------------------

\begin{document}

\maketitle

\tableofcontents

\section{Lagrangian formalism}

    From the start, one needs two distingue two types of field theories:
    \begin{itemize}
        \item Field theories with \emph{fixed background} where the metric is fixed (usually a solution of Einstein's equations in vacuum) and do not interact with the other fields. Even though it is not physically correct, it is a very convenient setup which is widely used.
        \item Field theories with \emph{dynamical background} where the metric is dynamical, the action then usually includes the Einstein-Hilbert action and the metric can be considered more like a generic field, which interacts with the others.
    \end{itemize}
    The two types will be given precise definitions in the following.

    \subsection{Field theories with fixed background}

        A \emph{classical field theory} with fixed background is the data $(M,g,E,S)$ where
        \begin{enumerate}
            \item $M$ is a smooth $n$-manifold provided with a Lorentzian metric $g$ that define a Lorentzian manifold $(M,g)$ called \emph{spacetime},
            \item $E\xrightarrow[]{\pi}M$ is (real or complex) vector bundle on $M$. The sections of this bundle are called \emph{dynamical fields}. The set of all possible fields is then $\F\equiv\Gamma(E)$ and is called the \emph{field configuration space}. The fiber of $E$ is called the \emph{target space} of the fields and denoted $\T$,
            \item $S$ is a map called the \emph{action} 
            \begin{equation}
                S:\left(
                \begin{array}{ccc}
                    \F & \longrightarrow & \R \\
                    \phi & \longmapsto & S[g;\phi]
                \end{array}
                \right)\qquad\text{ with }
            \end{equation}
            which is defined from a \emph{lagrangian}
            \begin{equation}
                \L:\left(
                \begin{array}{ccc}
                    \F & \longrightarrow & C^\infty(M,\R) \\
                    \phi & \longmapsto & \L[g;\phi]
                \end{array}
                \right)
            \end{equation}
            as
            \begin{equation}
                S[g;\phi]=\int_M~dV_g\L[g,\phi]
            \end{equation}
            where $dV_g$ is the (peudo-)riemannian volume form on $(M,g)$. The lagrangian must satisfy the property
            \begin{equation}
                \L[a^*g;a^*\phi]=\L[g;\phi]\circ a\label{eq:lagcdt}
            \end{equation}
            for any diffeomorphism $a:M\to M$.
        \end{enumerate}
        In practice the lagrangian often depends (in an inexplicit way or not) on the metric so we wrote this dependence explicitely, it will be useful later on. Let us emphasize the fact that we distinguish the metric from the dynamical fields although it can also be viewed as the section of a vector bundle. In this kind of field theory the metric is therefore not like any other generic field but is a fixed quantity given beforehand and necessary to define the theory. In more physical terms, we consider a fixed background.

        The field configuration space, i.e. the target space, can take different forms depending on the theory. The most elementary choices are
        \begin{itemize}
            \item $\T=\R$: theory of one real scalar field,
            \item $\T=\C$: theory of one complex scalar field,
            \item $\T=\R^4$: theory of one real vector field,
            \item $\T=\C^4$: theory of one complex vector field,
        \end{itemize}
        We can have an arbitrary number of any field type at the same time by taking product of the those spaces.

    \subsection{Field theories with dynamical background}

\section{Field transformations}

    What does it mean for a action to be invariant under diffeomorphisms? One could agree that any action is diffemorphism-invariant since the integration on manifolds is diffeomorphism-invariant by definition. Since we did not use the expression the lagrangian, this would imply that all actions are invariant under all global coordinate transformations and, maybe even more shockingly, that all actions are conformaly invariant. This manifestly false statement misses the fact that the action is not just an integral but rather a functional, i.e. a map $S:\F\to\R$ which associates to any field $\phi$ a real number. More precisely, the differnece between dynamical and non-dynamical fields (the metric) si very important here.

    First, we need to distinguish two types of field transformations. An \emph{external field transformation} is a diffeomorphism $a:M\to M$, so it acts on the spacetime manifold. On the other hand, an \emph{internal field transformation} is a diffeomorphism $A:\F\to\F$, so it acts on the field configuration space.\todo{Is this well-defined ?}

    \subsection{External field transformations}

        What does it mean for a action to be invariant under diffeomorphisms? One could agree that any action is diffemorphism-invariant since the integration on manifolds is diffeomorphism-invariant by definition. Since we did not use the expression the lagrangian, this would imply that all actions are invariant under all global coordinate transformations and, maybe even more shockingly, that all actions are conformaly invariant. This manifestly false statement misses the fact that the action is not just an integral but rather a functional, i.e. a map $S:\F\to\R$ which associates to any field $\phi$ a real number. More precisely, the differnece between dynamical and non-dynamical fields (the metric) si very important here.

        Let us consider an external transformation $a$, i.e. $a$ is a diffeomorphism from $M$ to $M$. Since the action takes fields as argument and that the fields are smooth maps on $M$, we can define a transformation of the action $S$ induced from the external transformation $a$. More precisely, this can be defined in three ways:
        \begin{itemize}
            \item \emph{passive transformation}: $S[g;\phi]\mapsto S[a^*g,\phi]$,
            \item \emph{active transformation}: $S[g;\phi]\mapsto S[g;a^*\phi]$,
            \item \emph{proper diffeomorphism}: $S[g;\phi]\mapsto S[a^*g,a^*\phi]$  See e.g. \textbf{Proposition 16.6} (d) in \cite{Lee_2012}.
        \end{itemize}
        Now it make sense that we always have $S[g;\phi]=S[a^*g,a^*\phi]$, regardless of the action, by definition of the integral on a manifold. This can be shown properly.
        \begin{prop}
            Any action is invariant under any proper diffeomorphism.
        \end{prop}
        \begin{proof}
            Since $\L[g;\phi]\in C^\infty(M,\R)$ and $dV_g\in\Omega^n(M)$, we have $\omega\equiv\L[g;\phi]dV_g\in\Omega^n(M)$, therefore $a^*\eta=(\L[g;\phi]\circ a) a^*(dV_g)$. By property of the riemannian volume form, we also have $a^*(dV_g)=dV_{a^*g}$. Consequently,
            \begin{align}
                S[a^*g,a^*\phi] &= \int_M dV_{a^*g}\L[a^*g;a^*\phi] \\
                &\stackrel{\label{eq:lagcdt}}{=} \int_M a^*dV_g~(\L[g;\phi]\circ a) \\
                &= \int_M a^*(dV_g\L[g;\phi]) \\
                &= \int_{a(M)} dV_g\L[g;\phi] \\
                &= S[g;\phi].
            \end{align}
        \end{proof}
        So proper diffeomorphisms are a pure mathematical redundancies. They have no physical interest; the invariance of a theory under proper diffeomorphism is just a consequence of the coordinate-independent setup of differential geometry.

        \begin{prop}
            Passive and active transformations only differ by a proper diffeomorphism.
        \end{prop}
        \begin{proof}
            From the previous definitions, we can see that
        \begin{equation}
            S[a^*g;\phi]\mapsto S[(a^{-1})^*(a^*g);(a^{-1})^*\phi]=S[g;(a^{-1})^*\phi].
        \end{equation}
        \end{proof}
        Since proper diffeomorphisms are trivial transformations, this means that active and passive transformations have the same physical relevance or, in other words, that they are equivalent: doing one or the other\footnote{If one takes the inverse diffeomorphism.} is strictly the same operation. In particular, we have invariance under some passive transformations if and only if we have invariance under the corresponding active transformations. In physics, we usually choose to act on the dynamical fields, this called the \emph{active point of view}. On the other hand, we could choose to act on the non-dynamical fields, this is the \emph{passive point of view}. This point of view is useful to develop an intuition about the relevance of only transforming the background metric or the fields, not the two at the same time, which gives a proper diffeomorphism. (Insert Pauolo's thought experiment).
        
        Note that all of this discussion can be generalized for lagrangian densities that depend on vector fields, $n$-forms or any $(p,q)$-tensor in general by appropriately transporting the quantities with $a$ or $a^{-1}$. In our case we have only used $a$ because forms as well a functions are naturally pulled back. For a general $(p,q)$-tensor, we have
        \begin{equation}
            T\mapsto (a^*T)(\omega_1,\dots,\omega_p,X_1,\dots,X_q)\equiv T(a^*\omega_1,\dots,a^*\omega_p,(a^{-1})_*X_1,\dots,(a^{-1})_*X_q).
        \end{equation}

    \subsection{Internal field transformations}

        For external field transformation, we saw that we were acting the fields (either the metric or the other fields depending on the active/passive point of view) by the means of pullbacks, which means that we really are acting on $M$, since
        \begin{equation}
            a^*\phi=\phi\circ a.
        \end{equation}
        Now, on the other hand, an \emph{internal field transformation} is a map that acts directly on the fields:
        \begin{equation}
            A:\left(
            \begin{array}{ccc}
                \F & \longrightarrow & \F \\
                \phi & \longmapsto & A(\phi)
            \end{array}\right).
        \end{equation}
        and that can be of any form. In general, it will be the action of a Lie group on the fibers of $E$.

        By definition of external transformation, we can see that they are, in particular, internal transformations:
        \begin{prop}
            Any external field transformation can be viewed as an internal field transformation.
        \end{prop}
        \begin{proof}
            It suffices to define the internal field transformation $A_a$ associated to the external field transformation $a$ as
            \begin{equation}
                A_a(\phi)\equiv(a^{-1})^*\phi=\phi\circ a^{-1}.
            \end{equation}
        \end{proof}
        The for acting with the inverse pullback will be motivated later.

    \subsection{Symmetries}

        An \emph{external symmetry} is external field transformation $a$ such that $S[g;a^*\phi]=S[g;\phi]$. Similarly, an \emph{internal symmetry} is internal field transformation $A$ such that $S[g;A(\phi)]=S[g;\phi]$.

        The distinction between external and internal field transformation might look like it is useless since external transformation are in particular internal transformations. However, the crucial point is that, even though we usually use the active point of view when we talk about external field transformations, we showed that it was equivalent to the active point of view, in which the metric is being acted upon. In particular, this implies that if an external field transformation $a:M\to M$ is an isometry, i.e. $a^*g=g$ then this transformation is automatically a external symmetry, by definition, and vice-versa:
        \begin{prop}
            An external field transformation $a$ is a symmetry if and only if it is an isometry of the spacetime.
        \end{prop}

    \subsection{Fields and representations of the isometry group}

        
        
        Since any external transformation $a$ on $M$ is equivalent to an internal transformation $A_a$ on $\F$, we can see that
        \begin{eqnarray}
            S[g;\phi]=S[a^*g;\phi]=S[g;(a^{-1})^*\phi]=S[g;A_a(\phi)]
        \end{eqnarray}
        so the internal transformation $A_a$ associated to an isometry $a$ is an internal symmetry. In other words, the group of isometries of $(M,g)$ is realized in some way on the field configuration space.
        \begin{prop}
            The internal field symmetry $A_a$ associated to external symmetries $a$ form a representation of $\ISO(M,g)$ on $\F$.
        \end{prop}
        \begin{proof}
            Let $a,b\in\ISO(M,g)$ and $A_a,A_b:\F\to\F$ be the associated internal field symmetries, then
            \begin{itemize}
                \item $(A_a\circ A_b)(\phi)=\phi\circ b^{-1}\circ a^{-1} = \phi\circ(a\circ b)^{-1} = A_{a\circ b}$,
                \item $A_{\text{id}} = \phi\circ\text{id}=\phi$
            \end{itemize}
            for all $\phi\in\F$.
        \end{proof}
        This motivates our choice to act with the inverse pullback rather than the pullback itself, as the product would be ill-defined.

        To find the irreducible representations of $\ISO(M,g)$ on $\F$, one must study its infinite-dimensional representation theory. This also constrain the fields, which is a crucial point: the fields living on a fixed spacetime must form an infinite-dimensional representation of the isometry group this spacetime.
    
\section{Examples}

    \begin{examp}
            Enlightened by this discussion, let us show that the action of a free massless scalar field on a flat background is invariant under dilatation. We have
            \begin{equation}
                S[\phi]=\int_{\R^D}\d^Dx~\eta^{\mu\nu}\p_\mu\phi\p_\nu\phi.
            \end{equation}
    \end{examp}

\pagebreak

\nocite{*}



\printbibliography

\end{document}